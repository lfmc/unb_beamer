%% Beamer presentation template by Luis F. M. Cury 28/12/2024

%% Compile using LuaLaTeX
%% We use LuaLaTex for the package fontspec that allows us to use a system font.
%% Otherwise, LaTex or XeLaTex compiler `should` work.
%% The command to use LuaLaTex is as follows. Synctex is optional.
%% latexmk -synctex=1 -interaction=nonstopmode -file-line-error -lualatex presentation.tex

\documentclass[aspectratio=169, 12pt]{beamer}

\usepackage[english]{babel}
\usepackage{xcolor}
\usepackage{graphicx}
\usepackage{amsmath}
\usepackage{booktabs}
\usepackage{bm}
\usepackage[backend=biber,language=english,style=abnt,justify,ittitles,giveninits=true]{biblatex}	% Citações padrão ABNT
\usepackage{csquotes}

%% Disable this package if you don't want to use LuaLaTex.
%% To use UnB font download it at http://marca.unb.br/arquivosdigitais/files/fontes/UnB_Office_v1.0.zip
%% and follow your OS's installation instructions.
%% Otherwise, comment "%" to use a standard latex font.
\usepackage{fontspec}
\setmainfont{UnB Office}

%% Use beamer configuration based on UnB official colors.
\usepackage{unbBeamer}

%% Numbered captions
\setbeamertemplate{caption}[numbered]

\addbibresource{biblio.bib}

\title[Small Title]{Presentation Title}
\subtitle{Optional subtitle}
\date{DD/MM/YYYY}
\author[Doe]{John Doe\inst{1}}
\institute[UoE]{\inst{1} Template Department\\ University of Examples}

%% Disable lower left-hand navigation symbols.
\beamertemplatenavigationsymbolsempty

\begin{document}

%% Uncomment the following line to put all bibliography regardless of it being 
%% individually cited.
%% \nocite{*}

\begin{frame}
    \titlepage
\end{frame}

\begin{frame}
    \frametitle{Presentation outline}
    \tableofcontents
\end{frame}

\section{Section \#1}

\begin{frame}
    \frametitle{Frame title}
    Blank frame example.
    This is a \alert{text with alert}.
\end{frame}

\section{Section \#2}

\begin{frame}
    \frametitle{Frame with columns}
    \begin{columns}[c]
        \begin{column}{0.49\textwidth}
            \begin{enumerate}
                \item Delta
                \item Echo
                \item Foxtrot
            \end{enumerate}
        \end{column}
        \begin{column}{0.49\textwidth}
            \begin{itemize}
                \item Alpha
                \item Bravo
                \item Charlie
            \end{itemize}
        \end{column}
    \end{columns}
\end{frame}

\begin{frame}
    \frametitle{Frame with figure}
    \begin{figure}
        \centering
        \includegraphics[height=0.7\textheight]{fig/square.eps}
        \caption{An empty square.}
        \label{fig:square}
    \end{figure}
\end{frame}

\begin{frame}
    \frametitle{Frame with table}
    \begin{table}
        \centering
        \caption{Fibonacci sequence.}
        \begin{tabular}{ccc}
            \toprule
            $n$ & $a_n$ \\
            \midrule
            $0$ & $1$ \\
            $1$ & $1$ \\
            $2$ & $2$ \\
            $3$ & $3$ \\
            \bottomrule
        \end{tabular}
        \label{tab:fibonacci_sequence}
    \end{table}
\end{frame}

\begin{frame}
    \frametitle{Frame with references}
    The table \ref{tab:fibonacci_sequence} shows the first 4 numbers of the Fibbonaci sequence.

    The figure \ref{fig:square} shows a square.

    According to \textcite{key} this is a Beamer template.

    This is a Beamer template \cite{key}.

\end{frame}

\begin{frame}[allowframebreaks]
    \printbibliography
\end{frame}

\end{document}

